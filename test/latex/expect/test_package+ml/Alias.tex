\section{Module \ocamlinlinecode{Alias}}\label{container-page-test+u+package+++ml-module-Alias}%
\label{container-page-test+u+package+++ml-module-Alias-module-Foo+u++u+X}\ocamlcodefragment{\ocamltag{keyword}{module} \hyperref[container-page-test+u+package+++ml-module-Alias-module-Foo+u++u+X]{\ocamlinlinecode{Foo\_\allowbreak{}\_\allowbreak{}X}}}\ocamlcodefragment{ : \ocamltag{keyword}{sig}}\begin{ocamlindent}\label{container-page-test+u+package+++ml-module-Alias-module-Foo+u++u+X-type-t}\ocamlcodefragment{\ocamltag{keyword}{type} t = \hyperref[xref-unresolved]{\ocamlinlinecode{int}}}\begin{ocamlindent}Module Foo\_\_X documentation. This should appear in the documentation for the alias to this module 'X'\end{ocamlindent}%
\medbreak
\end{ocamlindent}%
\ocamlcodefragment{\ocamltag{keyword}{end}}\\
\label{container-page-test+u+package+++ml-module-Alias-module-X}\ocamlcodefragment{\ocamltag{keyword}{module} \hyperref[container-page-test+u+package+++ml-module-Alias-module-X]{\ocamlinlinecode{X}}}\ocamlcodefragment{ : \ocamltag{keyword}{sig} .\allowbreak{}.\allowbreak{}.\allowbreak{} \ocamltag{keyword}{end}}\\

\section{Module \ocamlinlinecode{Alias.\allowbreak{}X}}\label{cpage-test+u+package+++ml-module-Alias-module-X}%
\label{cpage-test+u+package+++ml-module-Alias-module-X-type-t}\ocamlcodefragment{\ocamltag{keyword}{type} t = int}\begin{ocamlindent}Module Foo\_\_X documentation. This should appear in the documentation for the alias to this module 'X'\end{ocamlindent}%
\medbreak



